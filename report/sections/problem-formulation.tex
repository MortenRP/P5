\section{Problem formulation}
So far we have considered different aspects of the abstract problem of detecting traffic patterns. However, before we construct a solution, the problem should be well-defined such that a solution is clear. The problem is defined by the following problem formulation:
\\\\
\emph{Can an intelligent agent model the road network of Beijing to process and analyse raw GPS data, collected from a network of GPS-enabled vehicles, so as to detect traffic patterns indicating abnormal traffic flow such that a map of the traffic state of a road network can be constructed, that can be used to suggest alternative routes for drivers to save time and in general obtain better utilization of the road network?}
\\\\
Decomposing the problem formulation, we get the following questions:

\begin{itemize}
\item How can a road network be modelled?
\item How can data be collected and communicated between GPS-devices and the agent?
\item How can the agent analyse traffic data to detect traffic patterns?
\item Can such a system deliver time savings for drivers?
\item Can such a system deliver better utilisation of road networks?
\end{itemize}

With the problem formulation defined we can now construct a solution that answers the above questions, consequently answering the problem formulation itself.

% Prob1: Can traffic patterns be identified by analysing live as well as historical GPS-data of moving vehicles, giving a more understandable picture of the traffic situation, such that road networks can be better utilised?

% input: Raw GPS data from gps devices in vehicles
% problem domain model: directed weighted graph
% problem solution model: search / constraint satisfaction problem? Search problem?
% output: "Map" of traffic incidents and congestion, where road weights have changed according to these problems
% 