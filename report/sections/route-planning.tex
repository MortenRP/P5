\section{Route planning}
% What is this section?
In this section the process of finding a route will be describe.
% Why is this section here?
The aim of this section is to give a clear overview of what elements the route planning consist of.
% How follows from the subsections....
This process can be divided into sub process that are each responsible for an area of work.
% evt. Main problems adressed in this section

\subsection{Parallelized bi-directional A*} \label{algorithms}

In the coming section the math and algorithms used will be look at. Why they where chosen over other algorithms of there kinde.
% What is it?
Bi-directional graph search is in it's simplest form just a two normal Dijkstra’s where one starts from the start node and the other starts from the goal node.
% Why use it?
The reason for using this algorithm over other graph search is that the the time for finding a path is $O(b^{(n/2)})$ where a normal Dijkstra’s takes $O(b^{n})$. Bi-directional also have the benefit of working well when running in paralleled.


%The problem with using a bidirectional search is to get the two patch to meet up. This is where constructive heuristic is normally used, for this project the function deskribed in section \ref{graph} is used. The next problem is how a disition about whict way to go, this is the job of the cost function. The cost function is deskribed is section \ref{graph}.

% How do we use it?
By using bi-directional search there is one problem that needs to be address, this is will the search meet op before the whole graph is search. There are more then one way of solving this problem, in this project A* have been chosen for solving this problem. The A* makes sure that the search from the start node wakes a path towards the end node and the other way around for the search that starts from end node.

\subsubsection{Cost function}
% What is it?
% Why use it?
% How do we use it?
