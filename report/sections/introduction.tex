\chapter*{\markboth{Introduction}{Introduction}Introduction}\label{chap:introduction}
\addcontentsline{toc}{chapter}{Introduction}

Transportation and traffic problems are an issue in today's societies. It has been estimated that the total number of vehicles in the world will have increased by more than 150\% between 2002 and 2030\cite{dargay2007vehicle}, which will in all likelihood aggravate the current transportation problems. Furthermore, many of these problems become accentuated when notable events occur, such as concerts, holidays, and traffic accidents. A well-planned infrastructure can alleviate many of the traffic issues, but is expensive to change when requirements for traffic change. Therefore it is advantageous if traffic problems can be alleviated by better utilisation of existing road networks. Such an alleviation would be possible if drivers would plan their routes based on how the traffic situation usually is in their area. People usually have a good idea of such \emph{traffic patterns} in their local community, and as such it would be useful if those traffic patterns could be defined and used by many drivers, to take into account the local traffic situation. However, it is costly for people to report such patterns for others to use, and therefore it is interesting to explore the possibilities of automatising such a task. This project aims to explore the possibilities of applying the field of machine intelligence and distributed systems to design an intelligent agent that analyses traffic to find patterns, such that traffic problems can be alleviated. 

%There is a need for gaining a better understanding of traffic patterns, people have an easy time understating patterns. A lot of data that needs to be gather and then reasoned about in a smart and efficient manner. For people this would be a massive task to analyse so much data. This raises a lot of possibility for solving this using computers.
%\todo{write: how are networks / distributed systems relevant for this problem}

%This project aims to explore these problems, formalizes them, and proposes a solution that uses machine intelligence and distributed systems to alleviate them. This raises the question ''can the study of machine intelligence and distributed systems help in solving this increasing load on the road network?'' 

%This could be by smarter use of the road network or by making a driving plan e.g. so all people do not drive to or from work at the same time. These are just some possibly solutions ideas.

%Transportation problems arise frequently in areas with high population density. Examples that come to mind are people standing in line for services, traffic congestion on roads, and overcrowded public transportation. Many of these problems become accentuated when notable events occur, such as concerts, people going on holidays, and traffic accidents. Designing an infrastructure with these potential problems in mind can certainly minimize the troubles they cause. However, infrastructures are typically designed with a particular context in mind and consequently do not respond well to serious changes in the area, such as an increasing number of people. Costs are high when roads have to be expanded, changed, or constructed, and maintenance costs increase as roads more regularly need to be renewed when the everyday traffic in the area increases. For these reasons, it is advantageous to achieve better utilisation of road networks, such that costs of overburdened roads can be lowered as well as saving time for the people stuck in traffic congestions. This leads to an initial problem for this project: How can we analyse traffic to detect current and possibly future problems, such that initiatives for alleviating them can be considered.\todo{Læs intro. Måske nogle ting her, der tilhører relevans. Skal vi havde noget med hvordan det påvirker folk physik}
