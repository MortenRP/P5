\chapter{Introduction}
When numerous people are geographically located in the same area, transportation conflicts often arise. Examples that come to mind are people standing in line for services, traffic congestion on roads and overcrowded public transportation. Many of these problems become quite clear when extraordinary events happen, such as concerts, people going on holiday and traffic accidents. Designing infrastructure with these potential problems in mind, can certainly minimize the troubles they cause. However infrastructure is typically designed with a particular context in mind and consequently does not respond well to serious changes in the area, such as an increasing number of people. Costs are high when roads has to be expanded, changed or constructed and maintenance costs increase as roads more regularly needs to be renewed when the everyday traffic in the area increases. For these reasons, it is advantageous to achieve better utilisation of road networks, such that costs of overburdened roads can be lowered as well as saving time for the people stuck in traffic congestions. This leads to an initial problem for this project: How can we analyse traffic to detect current and possibly future problems, such that initiatives for alleviating them can be considered.\todo{Læs intro. Måske nogle ting her, der tilhører relevans}