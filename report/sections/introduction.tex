\chapter*{Introduction}\label{chap:introduction}

Transportation problems arise frequently in areas with high population density. It has been estimated that the total number of vehicles in the world will have increased by more than 150\% between 2002 and 2030\cite{dargay2007vehicle}, which will in all likelihood aggravate the current transportation problems. Many of these problems become accentuated when notable events occur, such as concerts, holidays, and traffic accidents.

There is a need for gaining a better understating of traffic patterns, people have an easy time understating patterns. A lot of data that needs to be gather and then reasoned about in a smart and efficient manner. For people this would be a massive task to analyze so much data. This raises a lot of possibility for solving this using computers.

This report explores these problems, formalizes them, and proposes a solution that uses machine intelligence(MI) and distributed systems to alleviate them.\todo{Rewrite with smoother mentions of MI and DSN.} This raises the question ``can the study of MI and distributed systems help in solving this increasing load on the road network?''. This could be by smarter use of the road network or by making a driving plan fx. so all people don't drive to or from work at the same time. These are just some possibly solutions ideas, but a deeper understanding of this two areas of computer science are need to know if this is possible.
\todo{Move some/all of the previous introduction to "Relevance" and/or merge with the current introduction.}

%Transportation problems arise frequently in areas with high population density. Examples that come to mind are people standing in line for services, traffic congestion on roads, and overcrowded public transportation. Many of these problems become accentuated when notable events occur, such as concerts, people going on holidays, and traffic accidents. Designing an infrastructure with these potential problems in mind can certainly minimize the troubles they cause. However, infrastructures are typically designed with a particular context in mind and consequently do not respond well to serious changes in the area, such as an increasing number of people. Costs are high when roads have to be expanded, changed, or constructed, and maintenance costs increase as roads more regularly need to be renewed when the everyday traffic in the area increases. For these reasons, it is advantageous to achieve better utilisation of road networks, such that costs of overburdened roads can be lowered as well as saving time for the people stuck in traffic congestions. This leads to an initial problem for this project: How can we analyse traffic to detect current and possibly future problems, such that initiatives for alleviating them can be considered.\todo{Læs intro. Måske nogle ting her, der tilhører relevans. Skal vi havde noget med hvordan det påvirker folk physik}
