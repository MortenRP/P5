\section{Knowledge representation}
In this section we describe what we are going to consider knowledge in the system and how we are going to represent that knowledge. More concretely, we consider how we model knowledge about the road network and traffic in Beijing. We consider the knowledge representation for several reasons. First of all, we want to make sure that the model adequately portrays the real roads and traffic of Beijing. Secondly, we want to be able to reason about this knowledge to later be able to find patterns in the traffic.
% WHAT is going to be in this section
% WHY are we bothering with this section?
% HOW follows from the subsections.

\subsection{Road network representation}
% Why and what?
To represent the road network, we must model what the road network consists of. A road network generally consists of roads and intersections of roads. Obviously there are many different types of roads, such as bridges, boulevards and one-way roads, which all have different properties related to how you can travel on them. Instead of differentiating between types of roads, we can see them as a single type of road, having attributes with different values. As such, to represent the road network we use a weighted, directed graph.\\
% How ?
Let G be a weighted directed graph $G(V,E)$, where $V$ is the set of nodes corresponding to intersections, $E$ is the set edges corresponding to road-segments between nodes. Each $e \in E$ is a tuple $e=(n_o, n_t, s)$ where $n_o, n_t \in N$ and represents the direction of a road from the origin intersection, $n_o$ to the target intersection $n_t$ and $s$ is the speed limit. The speed limit captures the most important difference between different types of roads, since we are only interested in \emph{how fast can i travel along this road}. Furthermore, two-way roads are represented as two edges, $e_1, e_2$ where $e_1 = (u, v)$ and $e_2=(v, u)$. Each edge, $e$, also has an associated cost, $c(e)$ where  $c: E \rightarrow \mathbb R_+$ is the cost function. Usually, the cost of an edge is used to choose between edges in search algorithms. Similarly, we construct our cost function for this purpose, as described in \ref{patterns:model-trees}
\\
Representing the road network as a graph is sufficient to capture the important attributes of roads and intersections, since the these attributes can be taken account for by adjusting the cost function to represent differences between different types of roads. 

\subsection{Traffic representation}\label{KR:traffic}
We must also consider how to represent knowledge about previous traffic in the Road Network. Since traffic changes over time, we consider this a seperate model than the model for the road network, as the traffic describes dynamic properties of the road network, whereas model of the road network represents the static properties.\\
From the Bejing datset, we can find many useful properties about the traffic. More specifically, we wish to capture the notion of a person observing the state of traffic when driving on a road at a specific time. An example would be, that a driver could observe that the traffic flows at a lower speed than the speed limit on the road to work in the morning. By collecting many of such \emph{observations} of drivers at different times, we build a large set of observations that can be used to reason about how the traffic situation is at certain time periods. To capture these properties, we define a set of \emph{observations} on road segments.\\
Let an \emph{observation}, $O$, be:
\begin{align*}
O = (T, D, W, S, E)
\end{align*}
where T is a encoding of a timestamp, D is the encoding of the date, W=\{Mon, Tue, Wed, Thu, Fri, Sat\} is the weekday when the observation was made. S is a non-negative real number representing the observed speed, and E is the road-segment on which the observation was made.\\

\subsubsection{Estimating observation speed}\label{KR:speed}
When instantiating observations, time, date, weekday and segment can easily be extracted directly from the mapmatched GPS data. However, speed must be computed from the difference in time and distance between GPS waypoints.
% How to estimate speed
