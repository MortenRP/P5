\section{Solution criteria}
%This section describes how we evaluate the developed solution for the problem formulation. The solution criteria will be the basis of the conclusion in \secref{ch:conclusion}. 
We propose the following criteria that the agent must fulfil, for it to be an acceptable solution to the problem:

\begin{itemize}
\item The agent must provide time-wise equivalent or better routes than generated by Google Maps.
\item The agent must avoid planning routes through congested areas, even if alternative road is a longer distance.
\item The agent must not chose routes that is disproportionately longer than a shortest path, to obtain minuscule time-savings. 

\item The agent must avoid adding traffic to congested areas, that is, it must avoid planning routes that either goes through existing congestions or creates congestion if the driver overloads the capacity of some road. 

\item Usage of agent must partially solve congested areas, that is, the more individuals utilising directions of the agent, the less congestion there is in the road network.
\end{itemize}
To evaluate the agent against the criteria of time savings, we simulate several individuals wanting to travel through the road network. We set up the simulation such that the shortest path and time of travel corresponds to a congested area that is discovered by the agent. We then calculate a route with Google Maps and with the agent respectively and compare the properties of these routes. If the route calculated by the agent is equivalent or faster and is within some reasonable limit of distance traveled, compared to the route calculated by Google Maps, we consider the route calculated by the agent to be acceptable.

To evaluate how the agent partially solves the congestion, we consider how many individuals on the road network that must take directions from the agent, before congestion patterns no longer occurs. To do this, we simulate travels of individuals on the road network and estimate the capacity  of congested roads. Then we increase incrementally the number of individuals taking directions of the agent by recalculating their routes. By doing this, we can monitor the correlation between number of individuals of the agent and congestion. Consequently we can how many users of the agent there must be, to solve the congestion problems in the city. If this amount of users does not exceed the number of users that utilize their smart-phones or tablets for directions, we consider the agent acceptable in terms of alleviating road networks of congestion.
%This section describes how the solution will be evaluated.
%e have chosen that the main criteria the routes generated by our solution, are to be evaluated up against, are a similar route generated by a WMS, such as Google Maps. We will thereby examine whenver the system can generate a better route based on the traffic patterns that we identify from the historical data. This means that we should be successful in navigating people around possible traffic congestions, and other slow moving traffic.
%The main solution criteria to be considered will be:

%\begin{itemize}
	%\item Is the average speed on the normal congested %road improved?
	%\item Is the cars using the navigation getting to %their destination in a optimal way, not having to drive a massive detour,
%	\item Is the traffic distributed in a good manner %over the whole road network
%	\item Is the average time spent in congestions %reduced?
%\end{itemize}


%A hard criteria for our solution is that it has to get the vehicle from it's source start address to it's destination address, using the existing road network, and it has to comply to the restrictions that the roads have.

%Besides that it is possible to look at some of the following criteria:
%\todo{How and why all these criteria}
%\todo{add criteria: reduction in emissions e.g. avoid stopping the vehicle}

%\todo{Review how we can measure if we obtain better road utilization}
