\section{Solution criteria}\label{chap:solutioncriteria}
%This section describes how we evaluate the developed solution for the problem formulation. The solution criteria will be the basis of the conclusion in \secref{ch:conclusion}.
We propose the following criteria that the system must fulfil to be considered an acceptable solution to the problem:

\begin{enumerate}
	\item The system must provide time-wise equivalent or better routes than routes from the test set.\label{crit:timesaving}
	\item The learned traffic patterns must reasonably reflect the actual traffic situation.\label{crit:cost}
	%\item The agent must avoid planning routes through congested areas, even if doing so increases the overall distance.
	%\item The agent must not choose routes that are disproportionately longer than a shortest path to obtain minuscule time-savings.\label{crit:distance}
	%\item The agent must avoid adding traffic to congested areas, that is, it must avoid planning routes that either go through existing congestions or create a congestion by overloading the capacity of other roads. \label{crit:overload}
	%\item Realistic usage of the agent must alleviate congestion, that is, a reasonable fraction of motorists using the agent should mitigate congestion.\label{crit:congestion}
	\item The system must retrieve relevant data for learning traffic patterns and provide route directions to users.\label{crit:sendmessages}
	\item The system must be scalable for usage by a large number of users. \label{crit:severalconnections}
\end{enumerate}

\todo{out deep criteria}


%We set up the simulation with a route that is the shortest path between two nodes, such that the route goes through an area that is known by the agent to be congested in a specific time interval. We then calculate a route with Google Maps and with the agent respectively and compare the properties of these routes. If the route calculated by the agent is equally fast or faster than the route calculated by Google Maps, we consider criterion \ref{crit:timesaving} met. If the distance travelled is not disproportionally longer than that of the Google Maps route, criterion \ref{crit:distance} is also met.%

%To check whether the agent meets criteria \ref{crit:overload} and \ref{crit:congestion} we consider how the number of users of the agent affects the severity of congestions. To do this, we simulate travels of motorists on the Beijing road network and estimate the capacity of congested roads. Then we increase the number of users of the agent incrementally by recalculating their routes. By doing so, we can estimate the correlation between the number of users of the agent and congestion. Consequently, we can approximate how many users of the agent there must be to best alleviate congestion problems. If something\todo{Find out exactly how to evaluate this.} then criterion \ref{crit:overload} is met. If this amount of users does not exceed the number of users that utilize their smart-phones or tablets for directions,\todo{Mention that this is what we meant by "realistic usage" in criterion \ref{crit:congestion}} we consider criterion \ref{crit:congestion} met.

%This section describes how the solution will be evaluated.
%e have chosen that the main criteria the routes generated by our solution, are to be evaluated up against, are a similar route generated by a WMS, such as Google Maps. We will thereby examine whenver the system can generate a better route based on the traffic patterns that we identify from the historical data. This means that we should be successful in navigating people around possible traffic congestions, and other slow moving traffic.
%The main solution criteria to be considered will be:

%\begin{itemize}
	%\item Is the average speed on the normal congested %road improved?
	%\item Is the cars using the navigation getting to %their destination in a optimal way, not having to drive a massive detour,
%	\item Is the traffic distributed in a good manner %over the whole road network
%	\item Is the average time spent in congestions %reduced?
%\end{itemize}


%A hard criteria for our solution is that it has to get the vehicle from it's source start address to it's destination address, using the existing road network, and it has to comply to the restrictions that the roads have.

%Besides that it is possible to look at some of the following criteria:
%\todo{How and why all these criteria}
%\todo{add criteria: reduction in emissions e.g. avoid stopping the vehicle}

%\todo{Review how we can measure if we obtain better road utilization}
