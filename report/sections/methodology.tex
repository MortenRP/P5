\section{Method}
%How will we solve the problem?
To solve the problem of traffic congestion, we process a large dataset of vehicle movements within a city, to derive useful properties of roads such as average speeds and number of vehicles at certain periods in time. We will use this information, to find \emph{traffic patterns} in the data, such that congestion can inferred from the history of the traffic on a given road (e.g. a main road to the university could possibly have congestion from Monday to Friday every morning from 07.00 to 09.00). We will use these patterns in a route planning process to find a potentially faster route than the shortest path, by avoid choosing a roads going through a congested area. Ultimately, the agent should be able to suggest a driver of a vehicle wanting to travel from point a to b, a faster route and users of the agent should be directed away from congested areas, such that the overall congestion of a city is smaller, thus utilising the road network better.

\subsection*{Data source \& collection}
Because of the limited time we have for this project, we use a GPS data set from Beijing from 10.000 taxies over a period of a week, as the historical data\cite{Tdrive}. If we had more time we would have collected our data on our own. A problem regarding this data is the short time span, since it does not cover holidays, or other special days, which are often at fault for massive traffical problems.\todo{consider moving problems with our method into an evalution section later in the report} We are also going to use a large data set from Beijing to simulate real-time data, this data was gathered over a period of 5 years, and include a testbase of 182 people\cite{Geolife}. Both of the data sets was published by Microsoft Research.

