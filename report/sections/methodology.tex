\section{Method}
\todo{Flyt det her + solution criteria efter problemformulering}
%How will we solve the problem?
We will process a large dataset of vehicle movements within a city, to derive useful properties of roads such as normal speeds and number of vehicles at certain periods in time. We will use this information, to find patterns in the data, such that congestion can be detected from the history of the traffic on a given road (e.g. a main road to the university has congestion from Monday to Friday every morning from 07.00 to 09.00). We will use these patterns in the route planning process to find a potentially faster route than the shortest path, by avoid choosing a route going through a congested area. Ultimately, the agent should be able to suggest a driver of a vehicle wanting to travel from point a to b, a faster route.

\subsection*{Data source \& collection}
We will be using a GPS data set from Beijing from 10.000 taxies over a period of a week, as our historical data\cite{Tdrive}. A problem regarding this data is the short time span, since it does not cover holidays, or other special days, which are often at fault for massive traffical problems.\todo{consider moving problems with our method into an evalution section later in the report} We are also going to use a large data set from Beijing to simulate real-time data, this data was gathered over a period of 5 years, and include a testbase of 182 people\cite{Geolife}. Both of the data sets was published by Microsoft Research.

