\section{Existing work}
There exists a vast ways of planning a route on different devices and there is  a lot of research in this field as well, where people are trying to figure out ways to improve the already existing technologies/technics to improve route planning. The earliest versions of GPS devices, which could aid you in finding a route , only considered the shortest path or the fastests path to your destination. 
The lastest years there has been focus on trying to predict other things like live traffic patterns and changing conditions on your route.
In the following chapter interesting existing technologies will be discussed and scientific articles about route planning will also be introduced and explained. Though it is necessary to first explain shortly what GPS(Global Positioning System) is and what it does, since it is the key technology for all route planning devices.

GPS consists of 3 key elements, namely the satellites in space, monitoring stations on earth and a person and his GPS receiver. The satellites are needed to pinpoint a location on earth, where you need atleast 4 satellites to accomplish that. 
There are 4 unmanned monitoring stations and one manned, the unmanned stations receive constant data from the satellites and they forward it to the manned station, which then corrects the data and afterwards sends the data back up to the satellites  again. The satellites transmits low power radio signals on different frequencies for different users.
http://www.tomtom.com/howdoesitwork/page.php?ID=5\&CID=2\&Language=1
The exact method to pinpoint your location with the 4 or more satellites is through trilateration. The way trilateration works is illustrated in figur(trilateration), though this is only in 2D, where in the real world it is 3D, but the point is the same except it is just spheres instead of circles.
http://www.tomtom.com/howdoesitwork/page.php?ID=8\&CID=2\&Language=1\&Lid=4


Google Maps developed by Google is a navigation system, which people can download to their smartphones. It has a feature where the suggested routes will be coloured green, yellow or red where that indicates respectively clear, slow-moving or heavily congested traffic. 
Google Maps creates the routes from historical data and live data which is sent by sensors and smartphones. The historical data includes information about what day it is and what time of the day it is, to be able to try to predict if there can occur traffic jams. The live data which are sent by sensors are placed by the government or private companies who gather traficin formation, where the live data from the smartphones are from people who are driving on the roads, reporting how fast they are moving on a particular road. The map is not hardcoded on your device, but depending on your route and your whereabouts, segments of a map is downloaded to your device.

A typical GPS, such as a TomTom works as explained above via satellites. So it locates your position and then locate the point you want to navigate to, then it computes a route on a map, which is usually on a SD card inside the TomTom. There is no historical data applied to generate the routes.
It also uses a technology called RDS-TMC(Radio Data System-Traffic Message Channel), which is a service that proves live-time traffic situations to your GPS. RDS-TMC works by if there is a situation on a road, can be a crash, bad weather, queues or something else, it transmits data about this to a central information centre, where then it is transmitted to a TMC information service provider	, which encodes the traffic info, from there it is transmitted to a FM radio broadcast where the information is then sent out in RDS signals that GPS receives and decode the RDS signals to a visual representation.

Another way of receiving changing traffic information is through your radio in your car, this is still a used method to inform people if there has been some crash or there is heavy traffic on some road. Though this is only to inform people and they usually do not give a second route to avoid the heave traffic, and the message is typically announced once.
