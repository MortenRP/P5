\section{Existing work}
The simpler way of receiving updates on traffic is through the radio in your car. This method is still used to inform people if there has been some incident or if there is heavy traffic on some road. Though this is only to inform people and they usually do not suggest alternative routes to avoid the heavy traffic. The message is reported in by a person and is typically announced once. \todo{Måske skal det her op i introduction}

There exists vast ways of planning a route on different devices and there is  a lot of research in this field as well, trying to figure out ways to improve the already existing technologies and techniques. The earliest versions of simple GPS-devices, which could aid in finding a route, only considered the shortest path or the fastest path to your destination.\todo{Source?}

The recent years there has been focus on trying to predict other things like live traffic patterns and changing conditions on your route.\todo{Source?}

In the following section relevant existing technologies and scientific articles will be explored so as to gain insight into the existing work of the problem domain.
%Though it is necessary to first explain shortly what GPS is and what it does, since it is the key technology for all route planning devices.

\subsection*{Dedicated GPS products}
A typical GPS, such as a TomTom device works by using satellites. It locates your position and then locate the point you want to navigate to, then it computes a route on a map, which is usually on a SD-card inside the TomTom device. There is no historical data applied to generate the routes.

It also uses a technology called RDS-TMC (Radio Data System-Traffic Message Channel), which is a service that provides live-time traffic updates to your GPS. RDS-TMC works such that if there is an incident on a road, such as a crash, bad weather or queues it transmits data about this to a central information centre, that further transmit the information to a TMC information service provider. The information service provider encodes the traffic information, and transmits it to a FM radio broadcast where the information is then sent out in RDS signals which the GPS-device receives and decodes to a visual representation.

\subsection*{Web Mapping Services}
%GPS consists of 3 key elements, namely the satellites in space, monitoring stations on earth and a person and his GPS receiver. The satellites are needed to pinpoint a location on earth, where you need atleast 4 satellites to accomplish that. 
%There are 4 unmanned monitoring stations and one manned, the unmanned stations receive constant data from the satellites and they forward it to the manned station, which then corrects the data and afterwards sends the data back up to the satellites  again. The satellites transmits low power radio signals on different frequencies for different users.
%http://www.tomtom.com/howdoesitwork/page.php?ID=5\&CID=2\&Language=1
%The exact method to pinpoint your location with the 4 or more satellites is through trilateration. The way trilateration works is illustrated in figur(trilateration), though this is only in 2D, where in the real world it is 3D, but the point is the same except it is just spheres instead of circles.
%http://www.tomtom.com/howdoesitwork/page.php?ID=8\&CID=2\&Language=1\&Lid=4
%
Google Maps developed by Google is a Web Mapping Service that people can download to their mobile devices such as smartphones. It has a feature where the suggested routes will be coloured green, yellow or red indicating respectively clear, slow-moving or heavily congested traffic.

Google Maps creates the routes from historical data and live data which is sent by sensors and smartphones\cite{Googleabout}. The historical data includes information about what day it is and what time of the day it is, to be able to try to predict if there can occur traffic jams. The live data which are sent by sensors are placed by the government or private companies who gather traffic information, where the live data from the smartphones are from people who are driving on the roads, reporting automatically how fast they are moving on a particular road. The map is not hardcoded on your device, but depending on your route and your whereabouts, segments of a map is downloaded to your device.


\subsection*{Intelligent Transportation Systems}
So far we have discussed the more or less chronological order of popularity of transportation systems technologies. However these devices rely on relatively simple route planning mechanisms. Now, we turn to a more advanced form of transportation system, namely the Intelligent Transportation Systems. An Intelligent Transportation System is a broad term, describing a system that provides traffic services, such that different kinds of users can better utilize transportation networks\todo{Mere eller mindre selvskreven definition, måske anvend definition fra ITS handbook}. This implies, that ITS are not necessarily 'intelligent', as to act and make decisions for the users, but can be more of a tool for users to assist in making smart choices.

\subsubsection*{Obtaining traffic data}
Any ITS must consider some data if it is to be of any real use for the user. Obtaining data is therefore one of the aspects of such a system. We distinguish between two methods of obtaining traffic data: road-based and vehicle-based data collection. 

Road-based methods places the responsibility of collection data on equipment installed on the roads, such that passing vehicles has no involvement in the collection. Road-based methods\cite{PIARC0} include technologies such as inductive loops buried beneath the roads to sense passing vehicles, and infra-red or ultrasonic sensors mounted on different entities along the roads. Inductive loops has the advantage of working well, regardless of the surrounding environment, however they are costly to implement and maintain in existing road networks, since they need to be buried beneath the roads. The sensors has the advantage of being less costly in implementation and maintenance, but may have problems operating under bad environment conditions such as snowy weather\cite{KamranHaas2007,PIARC0}.

The vehicle-based methods of collecting traffic data includes the vehicles in the responsibility of data collection. This means that vehicles might be equipped with devices that, either coorporates with some service or other equipment installed on the roads. Vehicle based methods include cell-tower triangulation of mobile devices, RFID-based identification of vehicles with sensors on the roads (or other vehicles in a peer-2-peer network) and GPS-enabled devices, communicating with a GPS service provider or satellite. Regardless of using a road-based or vehicle-based method, installing and maintaining equipment on roads for an ITS is expensive[1]. Considering the GPS-enabled devices approach is independent of other physical equipment installed in the area or on the roads, and the widespread popularity of GPS-enabled devices such as smart-phones, this approach seems like the most flexible and cost-efficient way of collecting traffic data.
% maybe- Mesh network of cars\todo{Cite Morten's source}

\subsubsection*{Analysing traffic data}
When data i readily available for an ITS, the system must perform some kind of analysis to discover useful information about transportation networks. Expecially, we are interested in the detection of road incidents such as car accidents and roadwork, and traffic congestion. Such abnormal events on a road network, can be defined as negative deviation from normal traffic flow on a given road. The normal traffic flow can be determined by collecting and processing historical data such as the speed of vehicles travelling on the road.


Someone\todo{insert source} proposes to partition road networks into road segments, such that information relevant for each segment is linked with the respective segment. This accommodates special cases of roads, such as highways, where some traffic data collected from the road is not necessarily relevant for some other part of the highway. An example would be the E45 highway, where the traffic on E45 in Northern Jylland might not be relevant for the traffic on E45 in Southern Jylland.
How can traffic data be analysed?
- Traffic states
- Road segmentation, city segmentation
- Statistical data about road segments 