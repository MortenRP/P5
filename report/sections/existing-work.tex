\section{Existing work}
The simple way of receiving changing traffic information is through the radio in your car. This method is still a used to inform people if there has been some incident or if there is heavy traffic on some road. Though this is only to inform people and they usually do not suggest alternative routes to avoid the heavy traffic, and the message is typically announced once. \todo{Får radioen information fra tippere på stedet?} \todo{Måske skal det her op i introduction}
There exists vast ways of planning a route on different devices and there is  a lot of research in this field as well, trying to figure out ways to improve the already existing technologies and techniques. The earliest versions of simple GPS-devices, which could aid you in finding a route , only considered the shortest path or the fastest path to your destination. 
The recent years there has been focus on trying to predict other things like live traffic patterns and changing conditions on your route.
In the following section relevant existing technologies and scientific articles will be explored so as to gain insight into the existing work of the problem domain.
%Though it is necessary to first explain shortly what GPS is and what it does, since it is the key technology for all route planning devices.
\subsection{Dedicated GPS products}
A typical GPS, such as a TomTom devoce works by using satellites. It locates your position and then locate the point you want to navigate to, then it computes a route on a map, which is usually on a SD-card inside the TomTom device. There is no historical data applied to generate the routes.
It also uses a technology called RDS-TMC (Radio Data System-Traffic Message Channel), which is a service that provides live-time traffic updates to your GPS. RDS-TMC works such that if there is an incident on a road, such as a crash, bad weather or queues it transmits data about this to a central information centre, that further transmit the information to a TMC information service provider. The information service provider encodes the traffic information, and transmits it to a FM radio broadcast where the information is then sent out in RDS signals which the GPS-device receives and decodes to a visual representation.

\subsection{Web Mapping Services}
%GPS consists of 3 key elements, namely the satellites in space, monitoring stations on earth and a person and his GPS receiver. The satellites are needed to pinpoint a location on earth, where you need atleast 4 satellites to accomplish that. 
%There are 4 unmanned monitoring stations and one manned, the unmanned stations receive constant data from the satellites and they forward it to the manned station, which then corrects the data and afterwards sends the data back up to the satellites  again. The satellites transmits low power radio signals on different frequencies for different users.
%http://www.tomtom.com/howdoesitwork/page.php?ID=5\&CID=2\&Language=1
%The exact method to pinpoint your location with the 4 or more satellites is through trilateration. The way trilateration works is illustrated in figur(trilateration), though this is only in 2D, where in the real world it is 3D, but the point is the same except it is just spheres instead of circles.
%http://www.tomtom.com/howdoesitwork/page.php?ID=8\&CID=2\&Language=1\&Lid=4
%
Google Maps developed by Google is a Web Mapping Service that people can download to their mobile devices such as smartphones. It has a feature where the suggested routes will be coloured green, yellow or red indicating respectively clear, slow-moving or heavily congested traffic. 
Google Maps creates the routes from historical data and live data which is sent by sensors and smartphones\todo{source}. The historical data includes information about what day it is and what time of the day it is, to be able to try to predict if there can occur traffic jams. The live data which are sent by sensors are placed by the government or private companies who gather traffic information, where the live data from the smartphones are from people who are driving on the roads, reporting\todo{Hvordan? Rapporterer enheden selv, eller skal man indtaste} how fast they are moving on a particular road. The map is not hardcoded on your device, but depending on your route and your whereabouts, segments of a map is downloaded to your device.