\section{Existing work}
Many different methods for helping drivers better plan their routes have been made.
The simplest way of receiving updates on traffic is through the radio in your car. This method is still used to inform people if there has been some incident or if there is heavy traffic on some road. Though this is only to inform people and they usually do not suggest alternative routes to avoid the heavy traffic. The message is reported in by a person and is typically announced once.
There are many different tools for planning a route on different devices and there is a lot of research in this field as well, trying to figure out ways to improve the already existing technologies and techniques. The earliest versions of simple GPS-devices, which could aid in finding a route, only considered the shortest path or the fastest path to your destination.\todo{Source?}

The recent years there has been focus on trying to predict other things like live traffic patterns and changing conditions on your route.\todo{Source?}

In the following section relevant existing technologies and scientific articles will be explored so as to gain insight into the existing work of the problem domain.
%Though it is necessary to first explain shortly what GPS is and what it does, since it is the key technology for all route planning devices.
\subsection*{Dedicated GPS products}
A typical GPS, such as a TomTom device works by using satellites. It locates your position and then locates the point you want to navigate to, then it computes a route on a map, which is usually on a SD-card inside the TomTom device. There is no historical data applied to generate the routes.

It also uses a technology called RDS-TMC (Radio Data System-Traffic Message Channel), which is a service that provides live-time traffic updates to your GPS. RDS-TMC works such that if there is an incident on a road, such as a crash, bad weather or queues it transmits data about this to a central information centre, that further transmit the information to a TMC information service provider. The information service provider encodes the traffic information, and transmits it to a FM radio broadcast where the information is then sent out in RDS signals which the GPS-device receives and decodes to a visual representation. The reporting of incidents or congestion is done manually people getting information from external sources. In Denmark the Highway Agency is responsible for manually sending out this information\cite{Vejdirektorat}. 

\subsection*{Web Mapping Services}
Web mapping services (WMS) are an alternative to the first GPS-devices. WMS include popular services such as Google Maps, OpenStreetMap and Apple's Maps. Most of these services can be downloaded to mobile devices such as smartphones or tablets and provide directions for the user. Google maps has a feature where the suggested routes will be coloured green, yellow or red indicating respectively clear, slow-moving or heavily congested traffic.
Google Maps also creates the routes from historical data and live data which is sent by sensors and smartphones\cite{Googleabout}. The historical data includes information about what day it is and what time of the day it is, to be able to try to predict if there can occur traffic jams. The live data which are sent by sensors are placed by the government or private companies who gather traffic information, whereas the live data from the smartphones are from people who are driving on the roads, reporting automatically how fast they are moving on a particular road. The map is not hardcoded on your device, but depending on your route and your whereabouts, segments of a map is downloaded to your device. Google Maps does not pick alternative routes for you, but they usually suggest two other routes and give an estimated driving time for each route\cite{Ncta}.
\todo{research: how do these services get informed about traffic/incidents/congestion?}
\todo{collect all methods for obtaining data to section below}

\subsection*{Intelligent Transportation Systems}\label{sec:ITS}
Another type of relevant system is the Intelligent Transportation Systems. The ITS handbook\cite{PIAR07} defines the ITS as:
\\\\
\emph{''ITS – Intelligent Transport Systems – is a generic term for the integrated application of
communications, control and information processing technologies to the transportation system. The
resultant benefits save lives, time, money, energy and the environment. The term “ITS” is flexible and
capable of being interpreted in a broad or narrow way.''}
\\\\
Per definition, ITS covers a wide area of system types, varying from traffic safety and security systems to traffic congestion relief systems. Consequently, ITS considers many different methods for acquisition, processing and analysis of traffic data and as such, is interesting to further investigate.

\subsubsection*{Data acquisition}
ITS must consider some data if they are to be of any real use for the user. Obtaining data is therefore one of the aspects of such systems. We distinguish between two methods of obtaining traffic data: road-based and vehicle-based data collection. 

Road-based methods places the responsibility of collection data on equipment installed on the roads, such that passing vehicles has no involvement in the collection. Road-based methods\cite{PIARC0} include technologies such as inductive loops buried beneath the roads to sense passing vehicles, and infra-red or ultrasonic sensors mounted on different entities along the roads. Inductive loops has the advantage of working well, regardless of the surrounding environment, however they are costly to implement and maintain in existing road networks, since they need to be buried beneath the roads. The sensors has the advantage of being less costly in implementation and maintenance, but may have problems operating under bad environment conditions such as snowy weather\cite{KamranHaas2007,PIARC0}.

The vehicle-based methods of collecting traffic data includes the vehicles in the responsibility of data collection. This means that vehicles might be equipped with devices that either cooperate with some service or other equipment installed on the roads. Vehicle based methods include cell-tower triangulation of mobile devices, RFID-based identification of vehicles with sensors on the roads (or other vehicles in a peer-2-peer network) and GPS-enabled devices, communicating with a GPS service provider or satellite. Regardless of using a road-based or vehicle-based method, installing and maintaining equipment on roads for an ITS is expensive\cite{KamranHaas2007}. Considering the GPS-enabled devices approach is independent of other physical equipment installed in the area or on the roads, and the widespread popularity of GPS-enabled devices such as smart-phones, this approach seems like the most flexible and cost-efficient way of collecting traffic data.
% maybe- Mesh network of cars\todo{Cite Morten's source}

\subsubsection*{Data processing}
When data is being acquired by various sources, an ITS usually performs some processing. When data is collected from different sources, the system performs data fusion to merge different knowledge about single entities. The data is usually stored in a database system, such as the data can be accessed and analysed when needed. During the processing, the ITS must also perform different kinds of data exchange, to accommodate the different schemas of storage and communication e.g. transforming a communications stream from a vehicle into a database record.\\ 
Another processing task is to perform map-matching of raw data. Map-matching is the process of mapping location measurements to a map. This is required because the location measurements often enough is not accurate, and therefore must be matched to a map by processing the data. An example is gps-samples that can be inaccurate depending on the satellite connection.

\subsubsection*{Data analysis}
When data is processed and readily available, the system must perform some kind of analysis to discover useful information about transportation networks. Such information could be about car accidents, roadwork, and traffic congestion. The information can then be utilised to act in the problem domain, to either automatically alleviate a problem or help improve decision making of drivers by giving useful suggestions.
\\\\
Whenever useful information has been derived, an ITS must distribute the information to the users of the system. Many different communication channels exist, such as message signs on the road, direct change in directions on gps-devices and in-vehicle devices.

\subsection{Other analysis methods}
Abnormal events on a road network, can be defined as negative deviation from normal traffic flow on a given road. The normal traffic flow can be determined by collecting and processing historical data such as the speed of vehicles travelling on the road. Kamran and Haas\cite{KamranHaas2007} proposes to partition road networks into road segments, such that information relevant for each segment is linked with the respective segment. This accommodates special cases of roads, such as highways, where some traffic data collected from the road is not necessarily relevant for some other part of the highway. An example would be the E45 highway, where the traffic on E45 in Northern Jutland might not be relevant for the traffic on E45 in Southern Jutland. By maintaining the average vehicle speed on road segments for multiple time-intervals, an agent could gain useful insight into how traffic flows at different times and can consequently deduce a set of traffic patterns indicating congestion.