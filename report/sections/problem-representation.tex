\section{Problem representation}
Before a solution for the problem formulation can be designed, the problem domain must be defined, so that a model for representing the problem can be proposed. This section describes the problem domain and the following model of the problem domain.

\subsection*{Domain}
We define the problem domain as the road network and the requirements of drivers of vehicles on the road network. A road network consists of roads and intersections. A road has the properties of max-speed, average speed, throughput of vehicles. Likewise, 

{vehicles, roads, intersections, drivers' requirements for destinations, 


% Problem specification (formulation) --> Problem representation (Directed weighted graph) --> solution (optimal route)
% problem domain: what we need to know about

\subsection*{Model}
To represent the problem domain we model the problem as a weighted, directed graph. Let G be a graph $G(N,E)$ where $E$ is the set of edges corresponding to road segments between intersections, $N$ is the set of nodes corresponding to intersections of roads. Each edge $e_i \in E$ has an associated cost, $c(e_i)$ where  $c: E \rightarrow \mathbb Z_+^*$ is the cost function. The cost of each edge describes the preference of taking the particular road corresponding to the edge. We want to design an algorithm for the cost function, that dynamically change the cost for each edge, as the flow of traffic in the road network changes, such that better routes dynamically can be determined based on the current traffic situation.

\subsection*{Constraints(?)}
Here we define the constraints on the model from the problemformulation.