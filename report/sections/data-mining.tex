\section{Data mining}
This section describes the process of analysing the data to find congestion patterns. We first describe the knowledge we are interested in and how to represent it, then we consider how patterns are defined.

\subsection{Knowledge representation}
Knowledge about the road network comes from knowledge about the state of traffic on a road at a given time e.g. a  person could know about morning traffic on popular roads from his experience driving on that road to work in the morning. We wish for the system to gain this kind of knowledge, and to do that we must analyse the data to find congestion patterns.\\
But before the data can be analysed, a representation of the knowledge must be devised. Thus to represent the knowledge about the state of traffic, the state of traffic itself must be defined. We propose the state of traffic on some road, to be determined by one or more \emph{observations} of the traffic on said road. An \emph{observation} on a road, is a measurement of speed driven by a vehicle, on a certain date and time. Therefore, let $O$ be the set of all observations of all roads R; that is, an observation $o \in O$ is one measurement of the speed of a particular road on a particular time and day:
\begin{align*}
O: R \rightarrow S \times T \times D \times M \times Y
\end{align*}
where $S = \mathbb R^{+}$, $T$ is a 24-hour formatted time-stamp of the when the observation was made, $D= \{mon, tue, wed, thu, fri, sat, sun\}$,\\ $M = \{jan, feb, mar, apr, may, jun, jul, aug, sep, oct, nov, dec\}$ and Y is the year of the observation. \\
The state of traffic for a road 

% 1. What is knowledge?	 
% 2. Which knowledge are we interested in ?
% 3. How are the knowledge defined?
% 4. H

Let the set of all roads in the road network be $R = \{MAX, Length\}$ where the domain of $MAX$ is defined by table x. 
Every $r \in R$ also has an associated set of observations, $O_r \subset O $. This subset can be further specified by the function:
\begin{align*}
\omega: O_r \times D \times T_{start} \times T_{end} \rightarrow O_r'
\end{align*}
where $O_r' \in O_r$ is the set of observations for a given road, r, on some day, D within some time interval between $T_{start},T_{end} \in T$.\\
To distinguish between different degrees of congestion, 
we define several classes of congestion as shown in table \ref{tab:congestion-classes}. 

\begin{table}[]
\centering
\caption{Classes of congestion.}
\label{tab:congestion-classes}
\begin{tabular}{l|l}
\textbf{Class} & \textbf{Description} \\ \hline
None           & \textgreater0.9      \\
Mild           & 0.9 \textgreater     \\
Heavy          & 0.5 \textgreater    
\end{tabular}
\end{table}
Different classes of congestion describes the state of traffic for a given road. The \emph{none} class of congestion indicates no congestion, \emph{mild} indicates a minor congestion, where the speed is slower than the speed-limit. The \emph{heavy} class of congestion indicates major congestion where traffic is slowed down by 50\% or more than the speed limit.\\
We use these classes to compute the probabilities of congestion on some road given a day and time.
Let the probability of some class of congestion, given a set of observations be:
\begin{align*}
P: C \times O_r' \rightarrow x \,|\, 0 < x < 1 \: and \: x \in \mathbb R
\end{align*}
where $C=\{none, mild, heavy\}$, $O_r'$ is some set of observations of $r$ and $P(none, O_r')+P(mild, O_r')+P(heavy, O_r') = 1$.
As such, for any given road we can compute the probabilities of the different classes of congestion. 