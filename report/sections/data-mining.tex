\section{Finding traffic patterns}
% Why find traffic patterns?
% What is this section for?
% How? --> follows from the subsections.
In order to reason about the traffic in Beijing, and provide some measure of how the previous traffic possibly affects our belief in how the future traffic situation will be in the road network, we analyse the preprocessed data to look for patterns. In this section, we consider methods for extracting such patterns and propose how to apply them on the Bejing dataset.
\subsection{Methods}\label{patterns:methods}
From our knowledge representation of the road network and traffic, we want to predict how traffic affects the travel time of roads in the road network. In order to do this, we need a measure of the traffic on roads. There are several ways one could construct such a measure, incuding
\begin{itemize}
\item \emph{capacity}: How many vehicles can travel on a road, safely, while not negatively affecting the average speed driven on the road?
\item \emph{classification}: Define a class feature of a road, that defines the state of the road. An example could be $Traffic \in \{free, light, medium, heavy\}$. A given road then has a classification based on the observations of speed within some time period.
\item \emph{speed}: we measure the traffic on a given road, by how fast you can drive on the road. In addition, one could measure the ratio between speed and speed limit.
\end{itemize}
The capacity of a road would be possible to estimate from the speed limit, length of a road, safety distances and number of cars. However, drivers do not always drive in a uniform manner, which makes it difficult to accurately estimate capacities.
Performing some discrete classification could help to distinguish between different traffic states. Intuitively, a person might be interested in knowing if a road is clear, mildy congested or heavyily congested. Such information could be captured by classifying roads by a roatio between speed limit and actual speed. This raises the questions of which classes should be used and how many. If there are not enough classes, it might be difficult to choose which road to take if both are congested more or less equally.
To more easily be able to evaluate roads where traffic is similar, we must use a measure that has a finer granularity. By looking directly at the actual speed driven on roads, we can determine more precisely the costs of taken a particular road. The actual speed, together with the length of a road is also the most important factors in in evaluating the travel time for a road. Therefore, we take a regression approach for predicting traffic.
\subsection{Model trees}\label{patterns:model-trees}
To predict the traffic speed on a given road segment, at a given time we can consider a prediction a function:
\begin{align*}
speed: S \times T \times W \rightarrow \mathbb{R}
\end{align*}
that takes a segment and time of the wanted prediction, and outputs the prediction as a real valued speed. The remaining question is then, how to actually perform the prediction of traffic on a segment given a specific time.
Predicting the numeric value of some variable based on previous observations is typically associated with regression. Regression fits input and output pairs to a function, such that future values for input pairs can be predicted. Fitting a regression function on the observations for a particular road segment, can provide a way to predict the traffic on the segment. Since traffic varies over time, one could argue that a polynomial regression function would fit the observations better than a linear function as illustrated in figure \ref{fig:compare-regression}.
\begin{figure}
\label{fig:compare-regression}
\includegraphics[width=\textwidth]{figures/compare-regression.pdf}
\caption{Different regression fits on the same dataset.}
\end{figure}
Even though the higher order regression functions generally fits the training data better, this might not be the case for future observations where the data actually might fit the linear case best. However, a linear regression over the observations for a whole day is not as useful for predicting the fluctuations in traffic during the day, because linear functions portrays straight lines, whereas fluctuations would be more acurately portrayed as curves in the function. For this reason, we instead utilise several linear regression functions over the course of a day. More specifically, we split a day  into partitions $p_1,...,p_i,...,p_n$ where $p_i = (t1,t2)$ and $t1,t2$ are time-stamps. Thus, a partition is defined as a period of time between $t1$ and $t2$. For every $p_i$, we perform a linear regression over the observations within that period of time. In practice, the overall regression function, $speed$, for a road segment becomes a piecewise linear function:
\[ speed(s,t) =
  \begin{cases}
    f_1(s)       & \quad \text{if } t \text{ is in } p_1\\
    f_2(s)  & \quad \text{if } t \text{ is in } p_2\\
    &\vdots\\
    f_n(s) & \quad \text{if } t \text{ is in } p_n
  \end{cases}
\]
where $f_1,...,f_n$ are linear regression functions for each partition over the observations for segment s. The idea is illustrated in figure \ref{fig:segmented-regression}
\begin{figure}\label{fig:segmented-regression}
\centering
TODO: DRAW FIGURE
\caption{Segmentation based linear regression}
\end{figure}
% Why use it?
% - We want to predict impact of traffic on travel time.
% - How do we measure traffic ? 
% 	-	Capacity of roads?
% 		- Difficult to determine accurately, people drive unsafe?
% 	- 	Speeds on roads?
% 	-	Try to classify on congestion?
% 		- k-ary class feature, that classifies a road.
% 		- Penalty of different classes.
% 		- How do we discretize our data into reasonable classes?
% 		- What about roads costs that are almost equal?
% 		- Classes defined by us might not reflect reality.
% 	-	Speed is more "pure" measurement of travel properties of a road
% 		- Speed tells both tells us the penalty of driving on road and has a better granularity of deciding between two almost equal costs of roads.
% 		- Methods for learning traffic:
% 			-	Linear regression
% 			-	Polynomial regression
% 				- Fits the learning data better, but might not fit future samples better! (SSE)
% 			-	But! Traffic also changes over time? (day, week etc..)
% 			-	Divide days into segments, for every segment define fit a linear regression function.
% 			-	How do we segment days?
% 			-	Residual sum
% 			-	Algorithms, M5P automatically splits and generates linear functions
% 
% What is it?
% How did we use it?
% Alternatives ?
\\
\subsection{Partitioning scheme}\label{patterns:segmentation}
% Why do it?
% What is it?
% How do we use it?

\subsubsection{Implementation}\label{patterns:weka}
% What is it? (M5P algorithm)
% Why use it? (does exactly what we need)
% How do we use it? (M5p in weka)

