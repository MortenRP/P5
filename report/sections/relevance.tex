\section{Relevance}
Estimates show that the number of cars in the world is increasing year by year.\todo{Cite source: http://wardsauto.com/ar/world\_vehicle\_population\_110815} This leads to an increase in traffic congestions, which in turn increases the need for more efficient traffic flow systems.

As mentioned in \ref{chap:introduction} there are several problems related to traffic congestion, such as:
\begin{itemize}
	\item Wear on heavily trafficated roads
	\item Pollution
	\item Waste of work hours
	\item Stress
\end{itemize}

If the traffic can be distributed more evenly on the whole road network, the overall wear on the roads will be less, and therefore the maintaining of these roads can be more efficiently spaced. This can lead to less congestions, Since these are often a consequence of roadwork, especially on heavily trafficated roads.\todo{@Kristian: fix this section}

One of the main things that are desired is to reduce the time wasted when waiting in line at a congestion.
The time wasted in a traffic jam is also time that the involved people could have used better, and since this time is also a stress factor for many, then there is a possibility for healthier people.

Pollution\todo{Write something about pollution}

The time spent waiting in line at a congestion, is time wasted. Congestions are often a mean of people getting to and from job, these can lead to stress for many of the drivers.\todo{Cite source: (stress-related) http://psycnet.apa.org/journals/apl/63/4/467.pdf} Which again can lead to frustrated drivers that begins to drive reckless.\todo{Cite source: http://www.sciencedirect.com/science/article/pii/S1369847899000029} Which can ultimately cause even greater problems in an already congested site. Even if the stress doesn't show as frustrated drivers it might have an impact on the drivers in other parts of their life.