\section{Relevance}
Traffic problems has several consequences and affects several different groups of people and organizations. Estimates show that the number of cars in the world is increasing year by year\cite{WardsAuto:CarPopulation}. This increase can potentially lead to an increase in traffic congestions, which in turn increases the need for better infrastructure or more efficient traffic flow. 
% wear on roads
Likewise, many of the roads that are congested are heavily used roads. These roads see more wear than less used roads and therefore requires either special construction or more maintenance, which increases the costs for governments on infrastructure. Maintenance of roads also obstructs the road network, since parts of a road can be unavailable which can result in temporarily slower traffic.
% Emissions
Another issue that is a consequence of traffic congestion is the increased CO2-emissions of vehicles continuously accelerating and de-accelerating in a congested area\cite{BarthBoriboonsomsin2009}.\\
% Time wasted --> stress
Being stuck in traffic can also affect the well-being of drivers as stress-levels increase. A study of the relationship between traffic congestion and driver stress observes that there is a correlation between drivers stressed behaviour and the traffic situation\cite{HennesyWiesenthal1997}\cite{StokolsNovacoStokolsCampell1978}. Stress has several negative impacts on the health of the stressed individual, which in worst case can lead to medical problems. Furthermore, severely stressed drivers can risk driving recklessly to get out of congestion, endangering themselves and others in traffic\cite{Shinar1998}. \\
% Emergency vehicles obstructed by congestion
Another critical issue is that emergency vehicles needing to travel through congested areas might be obstructed by traffic congestion. This is can at the worst case result in life-threatening situations which is why avoiding congestion is critical.
The economic consequences of traffic congestion are severe as the combined annual costs of congestion in U.S., U.K., France and Germany is estimated to increase to \$293.1 billion by 2030\cite{INRIX2013}. For		 individuals this translates into an annual, average cost of \$1740 and 111 hours wasted stuck in traffic.\\
To sum up the various issues of traffic congestion and whom they affect, \tabref{tab:relevance} shows a summary.

\begin{table}[]
\centering
\caption{Summary of issues involving interested parties}
\label{tab:relevance}
\begin{tabular}{lllll}
\multicolumn{1}{c|}{Interested party} & \multicolumn{1}{c}{Affected by}                                                                                                 &  &  &  \\ \cline{1-2}
\multicolumn{1}{l|}{Drivers}          & \begin{tabular}[c]{@{}l@{}}time wasted in traffic\\ stress from congestion\\ CO2-emissions\\ costs of congestion\end{tabular}   &  &  &  \\ \cline{1-2}
\multicolumn{1}{l|}{Governments}      & \begin{tabular}[c]{@{}l@{}}maintenance costs\\ emergency vehicle obstruction\\ CO2-emissions\\ costs of congestion\end{tabular} &  &  &  \\
                                      &                                                                                                                                 &  &  & 
\end{tabular}
\end{table}